%%%%%%%%%%%%%%%%%%%%%%%%%%%%%%%%%%%%%%%%%
% Classic Lined Title Page 
% LaTeX Template
% Version 1.0 (27/12/12)
%
% This template has been downloaded from:
% http://www.LaTeXTemplates.com
%
% Original author:
% Peter Wilson (herries.press@earthlink.net)
%
% License:
% CC BY-NC-SA 3.0 (http://creativecommons.org/licenses/by-nc-sa/3.0/)
% 
% Instructions for using this template:
% This title page compiles as is. If you wish to include this title page in 
% another document, you will need to copy everything before 
% \begin{document} into the preamble of your document. The title page is
% then included using \titleAT within your document.
%
%%%%%%%%%%%%%%%%%%%%%%%%%%%%%%%%%%%%%%%%%

%----------------------------------------------------------------------------------------
%	PACKAGES AND OTHER DOCUMENT CONFIGURATIONS
%----------------------------------------------------------------------------------------



\documentclass[a4paper]{article}

\usepackage[utf8]{inputenc}
\usepackage{hyperref}
\usepackage{titlesec}
\usepackage{amssymb,amsmath}
\usepackage[italian]{babel}
\usepackage[titles]{tocloft}
\usepackage{appendix}
\hypersetup{%
    pdfborder = {0 0 0}
}
\usepackage{graphicx}
\usepackage[svgnames]{xcolor} % Required to specify font color
\usepackage{eurosym}

\newcommand*{\plogo}{\fbox{$\mathcal{PL}$}} % Generic publisher logo
\usepackage{listings}
\usepackage{fancyhdr}
\usepackage{lastpage}
\usepackage{ragged2e}

%----------------------------------------------------------------------------------------
%	TITLE PAGE
%----------------------------------------------------------------------------------------

\newcommand*{\titleAT}{\begingroup % Create the command for including the title page in the document
\newlength{\drop} % Command for generating a specific amount of whitespace
\drop=0.1\textheight % Define the command as 10% of the total text height

\rule{\textwidth}{1pt}\par % Thick horizontal line
\vspace{2pt}\vspace{-\baselineskip} % Whitespace between lines
\rule{\textwidth}{0.4pt}\par % Thin horizontal line

\vspace{\drop} % Whitespace between the top lines and title
\begin{center} % Center all text
\textcolor{Black}{ % Red font color
{\Huge Algoritmi}\\[0.5\baselineskip] % Title line 1
{\Large e}\\[0.75\baselineskip] % Title line 2
{\Huge Strutture Dati}} % Title line 3
\\[2\baselineskip]
{\Large \version{}}

\vspace{0.25\drop} % Whitespace between the title and short horizontal line
\rule{1\textwidth}{0.4pt}\par % Short horizontal line under the title
\vspace{\drop} % Whitespace between the thin horizontal line and the author name

\end{center}
\vfill % Whitespace between the author name and publisher text
\vfill
{\large Luca De Franceschi}
\hfill
{\large Università degli studi di Padova}



%\rule{\textwidth}{0.4pt}\par % Thin horizontal line
%\vspace{2pt}\vspace{-\baselineskip} % Whitespace between lines
%\rule{\textwidth}{1pt}\par % Thick horizontal line

\endgroup}

\titleformat{\chapter}[display]
{}{\hfill\rule{.7\textwidth}{3pt}}{2pt}
{\hspace*{.3\textwidth}\huge\bfseries}[\addvspace{1pt}]
\titleformat{name=\chapter,numberless}[display]
{}{\hfill\rule{.7\textwidth}{3pt}}{2pt}
{\hspace*{.3\textwidth}\huge\bfseries}[\addvspace{1pt}]

\renewcommand*\contentsname{Indice}

\newcommand{\glossario}[1]{\textit{#1\ped{G}}}

\lstset{frame=shadowbox,
  aboveskip=3mm,
  belowskip=3mm,
  showstringspaces=false,
  columns=flexible,
  basicstyle={\small\ttfamily},
  numbers=none,
  numberstyle=\tiny\color{gray},
  keywordstyle=\color{blue},
  commentstyle=\color{dkgreen},
  stringstyle=\color{mauve},
  breaklines=true,
  breakatwhitespace=true
  tabsize=3
}

\renewcommand{\lstlistingname}{Listato}
\renewcommand{\footrulewidth}{0.4pt}

\newcommand{\changefont}{%
    \fontsize{5}{7}\selectfont
}

\def\arraystretch{2}

%----------------------------------------------------------------------------------------
%	CONSTANTS
%----------------------------------------------------------------------------------------

\newcommand{\version}{v0.3.0}

\newcommand{\authorName}{Luca De Franceschi}

%----------------------------------------------------------------------------------------
%	DOCUMENT HEADER
%----------------------------------------------------------------------------------------

\begin{document}
\justifying
\titleAT % This command includes the title page
\pagestyle{empty}
\newpage
\section*{Diario delle modifiche}

\begin{center}

	\begin{table}[htpd]
		\begin{tabular}{| l | l | l | p{50mm} |}
			\hline
			\hline
			\textbf{Autore} & \textbf{Versione} & \textbf{Data} & \textbf{Descrizione} \\
			\hline
			\hline
			Luca De Franceschi & 0.8.0 & 16/06/2014 & Inserito capitolo codici di Huffman \\ \hline
			Luca De Franceschi & 0.7.0 & 15/06/2014 & Inserito capitolo hashing e risposte a domande degli appelli \\ \hline
			Luca De Franceschi & 0.6.0 & 14/06/2014 & Inserito capitolo algoritmi golosi con esercizio del pulmino \\ \hline
			Luca De Franceschi & 0.5.0 & 13/06/2014 & Inserito capitolo analisi ammortizzata con due esercizi, incrementata sezione metodo dell'integrale \\ \hline
			Luca De Franceschi & 0.4.0 & 11/06/2014 & Inserito capitolo analisi complessità con metodo dell'integrale e metodo dell'esperto \\ \hline
			Luca De Franceschi & 0.3.0 & 11/06/2014 & Inserita spiegazione metodo di sostituzione \\ \hline
			Luca De Franceschi & 0.2.0 & 11/06/2014 & Inserita teoria su programmazione dinamica \\ \hline
			Luca De Franceschi & 0.1.0 & 11/06/2014 & Creata struttura del documento \\ \hline
		\end{tabular}
	\end{table}
	
\end{center}
\tableofcontents
\clearpage

%----------------------------------------------------------------------------------------
%	HEADER FORMAT
%----------------------------------------------------------------------------------------

\fancyhf{}
\fancyhead[RE]{\small\scshape\nouppercase{\leftmark}}
\fancyhead[LO]{\small\scshape\nouppercase{\rightmark}}
\fancyhead[LE,RO]{\small\thepage}
\lhead{\rightmark}
\rhead{\leftmark}
\rfoot{\thepage/\pageref{LastPage}}
\lfoot{\authorName}

\pagestyle{fancy}


%----------------------------------------------------------------------------------------
%	CONTENT
%----------------------------------------------------------------------------------------

\section{Risoluzione di ricorrenze}

\subsection{Metodo di sostituzione}

Per capire il metodo di sostituzione proviamo a risolvere il seguente esercizio:
\linebreak
\linebreak
La ricorrenza $T(n)=4T(n/2)+n^2\log n$ si può risolvere con il metodo dell'esperto? Giustificare la risposta. Se la risposta è negativa usare il metodo di sostituzione per dimostrare che $T(n)=O(n^2\log^2n)$.
\linebreak
\linebreak
Anzitutto vediamo i dati a disposizione:

$$a = 4, b = 2$$ 
$$f(n)=n^2\log n$$
$$g(n)=n^{\log_{b}a}=n^{\log_{2}4}=n^2$$

Calcoliamo ora il limite:

$$\lim_{n \to +\infty}\frac{n^2\log n}{n^2}=\infty$$

Da cui deduco che:

$$f(n)=\Omega(n^2)$$

Potrei dunque essere nel caso 3. Devo trovare un $$\epsilon > 0$$  tale che:

$$\lim_{n \to +\infty}\frac{n^2\log n}{n^{2+\epsilon}}\neq0$$

Ma mi accorgo subito che la cosa è impossibile, in quanto il denominatore, incrementando l'esponente, crescerà molto più velocemente rispetto al numeratore, per cui avrò sempre un valore tendente allo zero. Da questa considerazione deduco che la ricorrenza \textbf{non è risolvibile con il metodo dell'esperto}.
\linebreak
\linebreak
Procedo dunque con la sostituzione. Proviamo $T(n)=O(n^2\log^2n)$. Assumiamo che per un'opportuna costante $C>1$ e $\forall x<n$ sia verificata la disuguaglianza $T(x)\le C(x^2\log^2x)$ e dimostriamo che vale anche per $n$:

$$T(n)=4T(n/2)+n^2\log n \le 4C(n/2)^2\log^2(n/2)+n^2\log n$$
$$=Cn^2(\log n -1)^2+n^2\log n$$
$$=Cn^2(\log^2n-2\log n+1)+n^2\log n$$
$$=Cn^2\log^2n-2Cn^2\log n +Cn^2\log n+Cn^2+n^2\log n$$
$$=Cn^2\log^2n-(C-1)n^2\log n-Cn^2(\log n -1)$$

Ora applico una \textbf{maggiorazione}:

$$\le Cn^2\log^2n$$

Dunque ho dimostrato che: $T(n)=O(n^2\log^2n)$
\section{Programmazione dinamica}

In maniera del tutto generale la programmazione dinamica può essere descritta nel seguente modo:

\begin{enumerate}

\item Identifichiamo dei \textbf{sottoproblemi} del problema originario e utilizziamo una \textit{tabella} per memorizzare i risultati intermedi;
\item Inizialmente vanno definiti i \textbf{valori iniziali} di alcuni elementi della tabella, corrispondenti a sottoproblemi più semplici;
\item Al generico passo, avanziamo in modo opportuno sulla tabella calcolando il valore della soluzione di un sottoproblema in base alla soluzione di sottoproblemi precedentemente risolti (corrispondenti ad elementi della tabella precedentemente calcolati);
\item Alla fine restituiamo la soluzione del problema originario, che è stato memorizzato in un particolare elemento della tabella.

\end{enumerate}

La programmazione dinamica è usata normalmente per \textbf{problemi di ottimizzazione}, il termine ``programmazione'' si riferisce al metodo tabulare, non alla scrittura di codice.
\linebreak
\linebreak
La programmazione è applicabile con vantaggi se:

\begin{itemize}

\item Gode della proprietà di \textbf{sottostruttura ottima}: una soluzione si può costruire a partire da soluzioni ottime di sottoproblemi;
\item Il numero di sottoproblemi distinti è molto minore del numero di soluzioni possibili tra cui cercare quella ottima, altrimenti c'è la \textbf{ripetizione di sottoproblemi}, ovvero se il numero di sottoproblemi distinti è molto minore del numero di soluzioni possibili tra cui cercare quella ottima, allora uno stesso sottoproblema deve comparire molte volte come sottoproblema di altri sottoproblemi.

\end{itemize}

\subsection*{Ordine di calcolo delle soluzioni dei sottoproblemi}

\textbf{Bottom-up}: le soluzioni dei sottoproblemi del problema in esame sono già state calcolate. È il metodo migliore se per il calcolo della soluzione globale servono le soluzioni di tutti i sottoproblemi.
\linebreak
\linebreak
\textbf{Top-down}: è una procedura ricorsiva che dall'alto scende verso il basso. È la soluzione migliore se per il calcolo della soluzione globale servono soltanto alcune delle soluzioni dei sottoproblemi.

%\appendix

%\input{glossario}


\end{document}