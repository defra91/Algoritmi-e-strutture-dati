%%%%%%%%%%%%%%%%%%%%%%%%%%%%%%%%%%%%%%%%%
% Classic Lined Title Page 
% LaTeX Template
% Version 1.0 (27/12/12)
%
% This template has been downloaded from:
% http://www.LaTeXTemplates.com
%
% Original author:
% Peter Wilson (herries.press@earthlink.net)
%
% License:
% CC BY-NC-SA 3.0 (http://creativecommons.org/licenses/by-nc-sa/3.0/)
% 
% Instructions for using this template:
% This title page compiles as is. If you wish to include this title page in 
% another document, you will need to copy everything before 
% \begin{document} into the preamble of your document. The title page is
% then included using \titleAT within your document.
%
%%%%%%%%%%%%%%%%%%%%%%%%%%%%%%%%%%%%%%%%%

%----------------------------------------------------------------------------------------
%	PACKAGES AND OTHER DOCUMENT CONFIGURATIONS
%----------------------------------------------------------------------------------------



\documentclass[a4paper]{article}

\usepackage[utf8]{inputenc}
\usepackage{hyperref}
\usepackage{titlesec}
\usepackage{amssymb,amsmath}
\usepackage[italian]{babel}
\usepackage[titles]{tocloft}
\usepackage{appendix}
\hypersetup{%
    pdfborder = {0 0 0}
}
\usepackage{graphicx}
\usepackage[svgnames]{xcolor} % Required to specify font color
\usepackage{eurosym}

\newcommand*{\plogo}{\fbox{$\mathcal{PL}$}} % Generic publisher logo
\usepackage{listings}
\usepackage{fancyhdr}
\usepackage{lastpage}
\usepackage{ragged2e}

%----------------------------------------------------------------------------------------
%	TITLE PAGE
%----------------------------------------------------------------------------------------

\newcommand*{\titleAT}{\begingroup % Create the command for including the title page in the document
\newlength{\drop} % Command for generating a specific amount of whitespace
\drop=0.1\textheight % Define the command as 10% of the total text height

\rule{\textwidth}{1pt}\par % Thick horizontal line
\vspace{2pt}\vspace{-\baselineskip} % Whitespace between lines
\rule{\textwidth}{0.4pt}\par % Thin horizontal line

\vspace{\drop} % Whitespace between the top lines and title
\begin{center} % Center all text
\textcolor{Black}{ % Red font color
{\Huge Algoritmi}\\[0.5\baselineskip] % Title line 1
{\Large e}\\[0.75\baselineskip] % Title line 2
{\Huge Strutture Dati}} % Title line 3
\\[2\baselineskip]
{\Large \version{}}

\vspace{0.25\drop} % Whitespace between the title and short horizontal line
\rule{1\textwidth}{0.4pt}\par % Short horizontal line under the title
\vspace{\drop} % Whitespace between the thin horizontal line and the author name

\end{center}
\vfill % Whitespace between the author name and publisher text
\vfill
{\large Luca De Franceschi}
\hfill
{\large Università degli studi di Padova}



%\rule{\textwidth}{0.4pt}\par % Thin horizontal line
%\vspace{2pt}\vspace{-\baselineskip} % Whitespace between lines
%\rule{\textwidth}{1pt}\par % Thick horizontal line

\endgroup}

\titleformat{\chapter}[display]
{}{\hfill\rule{.7\textwidth}{3pt}}{2pt}
{\hspace*{.3\textwidth}\huge\bfseries}[\addvspace{1pt}]
\titleformat{name=\chapter,numberless}[display]
{}{\hfill\rule{.7\textwidth}{3pt}}{2pt}
{\hspace*{.3\textwidth}\huge\bfseries}[\addvspace{1pt}]

\renewcommand*\contentsname{Indice}

\newcommand{\glossario}[1]{\textit{#1\ped{G}}}

\lstset{frame=shadowbox,
  language=c++,
  aboveskip=10mm,
  belowskip=10mm,
  showstringspaces=false,
  columns=flexible,
  basicstyle={\small\ttfamily},
  numbers=none,
  numberstyle=\tiny\color{gray},
  keywordstyle=\color{blue},
  commentstyle=\color{gray},
  stringstyle=\color{red},
  breaklines=true,
  breakatwhitespace=true
  tabsize=1
}

\renewcommand{\lstlistingname}{Listato}
\renewcommand{\footrulewidth}{0.4pt}

\newcommand{\changefont}{%
    \fontsize{5}{7}\selectfont
}

\def\arraystretch{2}

%----------------------------------------------------------------------------------------
%	CONSTANTS
%----------------------------------------------------------------------------------------

\newcommand{\version}{v0.3.0}

\newcommand{\authorName}{Luca De Franceschi}

%----------------------------------------------------------------------------------------
%	DOCUMENT HEADER
%----------------------------------------------------------------------------------------

\begin{document}
\titleAT % This command includes the title page
\pagestyle{empty}
\newpage
\section*{Diario delle modifiche}

\begin{center}

	\begin{table}[htpd]
		\begin{tabular}{| l | l | l | p{50mm} |}
			\hline
			\hline
			\textbf{Autore} & \textbf{Versione} & \textbf{Data} & \textbf{Descrizione} \\
			\hline
			\hline
			Luca De Franceschi & 0.8.0 & 16/06/2014 & Inserito capitolo codici di Huffman \\ \hline
			Luca De Franceschi & 0.7.0 & 15/06/2014 & Inserito capitolo hashing e risposte a domande degli appelli \\ \hline
			Luca De Franceschi & 0.6.0 & 14/06/2014 & Inserito capitolo algoritmi golosi con esercizio del pulmino \\ \hline
			Luca De Franceschi & 0.5.0 & 13/06/2014 & Inserito capitolo analisi ammortizzata con due esercizi, incrementata sezione metodo dell'integrale \\ \hline
			Luca De Franceschi & 0.4.0 & 11/06/2014 & Inserito capitolo analisi complessità con metodo dell'integrale e metodo dell'esperto \\ \hline
			Luca De Franceschi & 0.3.0 & 11/06/2014 & Inserita spiegazione metodo di sostituzione \\ \hline
			Luca De Franceschi & 0.2.0 & 11/06/2014 & Inserita teoria su programmazione dinamica \\ \hline
			Luca De Franceschi & 0.1.0 & 11/06/2014 & Creata struttura del documento \\ \hline
		\end{tabular}
	\end{table}
	
\end{center}
\tableofcontents
\clearpage

%----------------------------------------------------------------------------------------
%	HEADER FORMAT
%----------------------------------------------------------------------------------------

\fancyhf{}
\fancyhead[RE]{\small\scshape\nouppercase{\leftmark}}
\fancyhead[LO]{\small\scshape\nouppercase{\rightmark}}
\fancyhead[LE,RO]{\small\thepage}
\lhead{\rightmark}
\rhead{Algoritmi e strutture dati}
\rfoot{\thepage/\pageref{LastPage}}
\lfoot{\authorName}

\pagestyle{fancy}


%----------------------------------------------------------------------------------------
%	CONTENT
%----------------------------------------------------------------------------------------

\raggedright

\section{Stima della complessità di un algoritmo}

Identifichiamo dei casi base, studiando la complessità degli algoritmi noti.

\begin{enumerate}

\item Le operazioni elementari, messe al di fuori dei cicli, e che riguardano l'uso di variabili hanno complessità costante $c_i$, aprossimabile a 0 nello studio della complessità asintotica;

\item Da \textit{insertion-sort} si vede che un \textit{for i = 2 to n} ha complessità pari a $n$. Constatiamo dunque che un ciclo for che va dall'indice $1$ all'indice $n$ avrà complessità $c_i(n+1)$;

\item Tutte le operazioni elementari che compaiono all'interno di un ciclo for di complessità $c_i(n+1)$ hanno complessità $c_in$;

\item Per i cicli annidati in altri cicli a complessità è data da $c_i\sum_{j=x}^{n}(c_j)$, dove gli estremi della sommatoria sono gli estremi del del ciclo esterno.

\end{enumerate}

Per valutare la complessità si scrive l'equazione $T(n)$ sommando tutte le complessità. Per studiare le sommatorie si utilizza il \textbf{metodo dell'integrale}.

\subsection{Metodo dell'integrale}

Se $f(x)$ è una funzione \textbf{non decrescente}:

$$\int_a^{b+1} f(x)\mathrm{d}x\le \sum_{i=a}^b f(i) \le \int_{a-1}^{b+1} f(x)\mathrm{d}x$$

Se $f(x)$ è una funzione \textbf{non decrescente}:

$$\int_a^{b+1} f(x)\mathrm{d}x \le \sum_{i=a}^b f(i) \le \int_{a-1}^b f(x)\mathrm{d}x$$

Inoltre riportiamo di seguito le comuni sommatorie:

\begin{itemize}

\item \textbf{Serie aritmetica}: $\sum_{i=1}^n i = \frac{n(n+1)}{2}$;
\item \textbf{Serie geometrica}: $\sum_{i=0}^k q^i = \frac{q^{k+1}-1}{q-1}$ \hfill $q\neq1$

\end{itemize}

\subsection{Andamento asintotico}

Una volta ottenuta una funzione che rappresenta la complessità dell'algoritmo ci può interessare prendere in esame l'andamento asintotico della medesima. Per farlo introduciamo le seguenti notazioni:

\begin{itemize}

\item \textbf{``$O$'' grande}: date due funzioni $f(n)$ e $g(n)$ si dice che $f(n)$ è ``$O$'' grande di $g(n)$ se esiste un $c>0$ e un $h_0$ tali che: 

$$f(n)\le cg(n)$$ 

per $n\ge h_0$ (\textbf{limite asintotico superiore}). In pratica l'ordine di crescita di $f(n)$ è non superiore a quello di $g(n)$;

\item \textbf{``$\Omega$'' grande}: date $f(n)$ e $g(n)$ si dice che $f(n)$ è ``$\Omega$'' grande di $g(n)$ se esiste una costante $c>0$ e un $h_0$ tale che:

$$f(n)\ge cg(n)$$ 

per $n\ge h_0$ (\textbf{limite asintotico inferiore}). In pratica l'ordine di crescita di $f(n)$ è non inferiore a quello di $g(n)$;

\item \textbf{``$\Theta$'' grande}: date $f(n)$ e $g(n)$ si dice che $f(n)$ è ``$\Theta$'' grande di $g(n)$ se ci sono costanti positive $c_1$, $c_2$ e un $h_0$ tali che:

$$c_1g(n)\le f(n) \le c_2g(n)$$

per $n\ge h_0$ (\textbf{limite asintotico stretto}). In pratica diciamo che se $f(n)=O(g(n))$ e $f(n)=\Omega(g(n))$ allora è vero anche che $f(n)=\Theta(g(n))$

\end{itemize}

Di una funzione non ci interessa la sua forma ma il suo comportamento asintotico. Spesso è possibile determinare dei limiti asintotici calcolando il limite di un rapporto:

$$\lim_{n \to \infty} \frac{f(n)}{g(n)}$$

In base al risultato di questo limite ho tre casi:

\begin{enumerate}

\item Ottengo un valore costante $k>0$: in questo caso $f(n)$ è dello stesso ordine di $g(n)$, e dunque:

$$\forall \epsilon>0, \exists h_0 | h\ge h_0 : k-\epsilon \le f(n)/g(n)\le k+\epsilon$$

ponendo:

$$c_1g(n)\le f(n) \le c_2g(n)$$

Dunque concludo dicendo che $f(n)=\Theta(g(n))$;

\item Il limite tende a $\infty$: $f(n)=\Omega(g(n))$;
\item Il limite tende a $0$: $f(n)=O(g(n))$.

\end{enumerate}

\subsection{Metodo dell'esperto}

Per risolvere le ricorrenze il primo metodo da utilizzare è il \textbf{metodo dell'esperto}. Se la ricorrenza è espressa nella forma:

$$T(n)=aT(n/b)+f(n)$$

e se $a \ge 1$ e $b<1$ allora:

\begin{enumerate}

\item Tolgo eventuali arrotondamenti;
\item Calcolo $\log_ba$ e calcolo il limite: $\lim_{n \to \infty}\frac{f(n)}{n^{\log_ba}}$;

\end{enumerate}

A questo punto, in base al valore del limite ho 3 possibili casi:

\subsubsection{Caso 2}

Se il limite è \textbf{finito} e diverso da zero:

$$f(n)=\Theta(n^{\log_ba})\Rightarrow T(n)=\Theta(n^{\log_ba}\log n)$$

\subsubsection{Caso 1}

Se il limite è \textbf{uguale a zero} devo trovare un valore $\epsilon > 0$ per il quale risulta finito il limite:

$$\lim_{n \to \infty}\frac{f(n)}{n^{\log_ba-\epsilon}}=k$$

Se lo trovo allora posso affermare che:

$$f(n)=O(n^{\log_ba-\epsilon}) \Rightarrow T(n)=\Theta(n^{\log_ba})$$

\subsubsection{Caso 3}

Se il limite è $\int$ allora devo trovare un $\epsilon >0$ per il quale risulti:

$$\lim_{n \to \infty}\frac{f(n)}{n^{\log_ba+\epsilon}}\neq 0$$

Se lo trovo allora devo studiare l'equazione:

$$af(n/b)\le k(f(n))$$

se trovo un $k<1$ allora posso concludere che:

$$f(n)=\Omega(n^{\log_ba+\epsilon})\Rightarrow T(n)=\Theta(f(n))$$

\subsection{Metodo di sostituzione}

Se non riesco ad applicare il metodo dell'esperto allora devo utilizzare il \textbf{metodo di sostituzione}.

Per capire il metodo di sostituzione proviamo a risolvere il seguente esercizio:
\linebreak
\linebreak
La ricorrenza $T(n)=4T(n/2)+n^2\log n$ si può risolvere con il metodo dell'esperto? Giustificare la risposta. Se la risposta è negativa usare il metodo di sostituzione per dimostrare che $T(n)=O(n^2\log^2n)$.
\linebreak
\linebreak
Anzitutto vediamo i dati a disposizione:

$$a = 4, b = 2$$ 
$$f(n)=n^2\log n$$
$$g(n)=n^{\log_{b}a}=n^{\log_{2}4}=n^2$$

Calcoliamo ora il limite:

$$\lim_{n \to +\infty}\frac{n^2\log n}{n^2}=\infty$$

Da cui deduco che:

$$f(n)=\Omega(n^2)$$

Potrei dunque essere nel caso 3. Devo trovare un $$\epsilon > 0$$  tale che:

$$\lim_{n \to +\infty}\frac{n^2\log n}{n^{2+\epsilon}}\neq0$$

Ma mi accorgo subito che la cosa è impossibile, in quanto il denominatore, incrementando l'esponente, crescerà molto più velocemente rispetto al numeratore, per cui avrò sempre un valore tendente allo zero. Da questa considerazione deduco che la ricorrenza \textbf{non è risolvibile con il metodo dell'esperto}.
\linebreak
\linebreak
Procedo dunque con la sostituzione. Proviamo $T(n)=O(n^2\log^2n)$. Assumiamo che per un'opportuna costante $C>1$ e $\forall x<n$ sia verificata la disuguaglianza $T(x)\le C(x^2\log^2x)$ e dimostriamo che vale anche per $n$:

$$T(n)=4T(n/2)+n^2\log n \le 4C(n/2)^2\log^2(n/2)+n^2\log n$$
$$=Cn^2(\log n -1)^2+n^2\log n$$
$$=Cn^2(\log^2n-2\log n+1)+n^2\log n$$
$$=Cn^2\log^2n-2Cn^2\log n +Cn^2\log n+Cn^2+n^2\log n$$
$$=Cn^2\log^2n-(C-1)n^2\log n-Cn^2(\log n -1)$$

Ora applico una \textbf{maggiorazione}:

$$\le Cn^2\log^2n$$

Dunque ho dimostrato che: $T(n)=O(n^2\log^2n)$
\section{Programmazione dinamica}

In maniera del tutto generale la programmazione dinamica può essere descritta nel seguente modo:

\begin{enumerate}

\item Identifichiamo dei \textbf{sottoproblemi} del problema originario e utilizziamo una \textit{tabella} per memorizzare i risultati intermedi;
\item Inizialmente vanno definiti i \textbf{valori iniziali} di alcuni elementi della tabella, corrispondenti a sottoproblemi più semplici;
\item Al generico passo, avanziamo in modo opportuno sulla tabella calcolando il valore della soluzione di un sottoproblema in base alla soluzione di sottoproblemi precedentemente risolti (corrispondenti ad elementi della tabella precedentemente calcolati);
\item Alla fine restituiamo la soluzione del problema originario, che è stato memorizzato in un particolare elemento della tabella.

\end{enumerate}

La programmazione dinamica è usata normalmente per \textbf{problemi di ottimizzazione}, il termine ``programmazione'' si riferisce al metodo tabulare, non alla scrittura di codice.
\linebreak
\linebreak
La programmazione è applicabile con vantaggi se:

\begin{itemize}

\item Gode della proprietà di \textbf{sottostruttura ottima}: una soluzione si può costruire a partire da soluzioni ottime di sottoproblemi;
\item Il numero di sottoproblemi distinti è molto minore del numero di soluzioni possibili tra cui cercare quella ottima, altrimenti c'è la \textbf{ripetizione di sottoproblemi}, ovvero se il numero di sottoproblemi distinti è molto minore del numero di soluzioni possibili tra cui cercare quella ottima, allora uno stesso sottoproblema deve comparire molte volte come sottoproblema di altri sottoproblemi.

\end{itemize}

\subsection*{Ordine di calcolo delle soluzioni dei sottoproblemi}

\textbf{Bottom-up}: le soluzioni dei sottoproblemi del problema in esame sono già state calcolate. È il metodo migliore se per il calcolo della soluzione globale servono le soluzioni di tutti i sottoproblemi.
\linebreak
\linebreak
\textbf{Top-down}: è una procedura ricorsiva che dall'alto scende verso il basso. È la soluzione migliore se per il calcolo della soluzione globale servono soltanto alcune delle soluzioni dei sottoproblemi.
\section{Analisi ammortizzata}

Si consideri il tempo richiesto per eseguire, nel \textbf{caso pessimo}, un'intera sequenza di operazioni. Se le operazioni costose sono relativamente meno frequenti allora il costo richiesto per eseguirle può essere ammortizzato con l'esecuzione delle operazioni meno costose.

\subsection{Metodo dell'aggregazione}

Si basa sul concetto di \textbf{costo ammortizzato}: data una sequenza di $n$ istruzioni aventi complessità $O(f(n))$, il costo ammortizzato della singola operazione si ottiene dividendo la complessità totale per il numero di istruzioni.

$$\hat{c}=\frac{O(f(n))}{n}$$

\subsection{Metodo degli accantonamenti (o dei crediti)}

Si caricano le operazioni meno costose di un costo aggiuntivo. Il costo aggiuntivo viene assegnato come \textbf{credito prepagato} a certi oggetti nella struttura dati. I crediti accumulati saranno usati per pagare le operazioni più costose su tali oggetti. 

Il costo ammortizzato delle operazioni meno costose è il costo effettivo aumentato del costo aggiuntivo. 

Il costo ammortizzato delle operazioni più costose è il costo effettivo diminuito del credito prepagato utilizzato.

Il costo artificiale fornisce un \textbf{limite superiore} al costo ammortizzato.

\subsection{Metodo del potenziale}

Si associa alla struttura dati $D$ un \textbf{potenziale} $\Phi(D)$ tale che la modifica della struttura dati dovuta alle operazioni meno costose comporti un aumento del potenziale, mentre le operazioni meno costose lo facciano diminuire.

Il costo ammortizzato è quindi la \textbf{somma algebrica} del costo effettivo e della variazione di potenziale. $D_i$ è la struttura dati dopo la \textit{c}-esima operazione e $c_i$ + il costo dell'\textit{i}-esima operazione.

$$\hat{c}_i=c_i+\Phi(D_i)-\Phi(D_{i-1})$$

Il costo ammortizzato di una sequenza di $n$ operazioni è:

$$\hat{c}=\sum_{i=1}^n\hat{c}_i=\sum_{i=1}^n[c_i+\Phi(D_i)-\Phi(D_{i-1})]$$
$$=c+\Phi(D_n)-\Phi(D_0)$$

Se la variazione $\Phi(D_n)-\Phi(D_0)$ del potenziale relativo all'esecuzione di tutta la sequenza non è negativa allora il costo ammortizzato $\hat{c}$ è una \textbf{maggiorazione} del costo reale $c$.

Altrimenti un valore $\Phi(D_n)-\Phi(D_0)$ negativo deve essere compensato con un aumento adeguato del costo ammortizzato delle operazioni.

\subsection{Esercizio 1}

Si vuole realizzare un contatore binario usando un array $A[0...k]$ per memorizzare i $k+1$ bit $b_k...b_1b_0$ della rappresentazione binaria del valore $x$ del contatore. Si vuole che il contatore possa iniziare anche con un valore maggiore di zero. A tale scopop si vuole che, oltre all'operazione \textbf{increment}, che aumenta di 1 il valore del contatore, sia prevista anche un'operazione iniziale \textbf{SET(A,n)}, che inizializza ad n il valore del contatore.
Usare il \textbf{metodo di aggregazione} per dimostrare che le operazioni di una sequenza costituita da una set seguita da $m$ increment, con $m=\Omega(k)$ hanno costo ammortizzato costante.
\linebreak
\linebreak
\textbf{Soluzione}:Vediamo anzitutto gli pseudo-codici delle due funzioni:

\begin{lstlisting}

Increment(A)
	i = 0
	while i<=k and A[i]==1
		A[i] = 0
		i++
	if i<=k
		A[i] = 1

\end{lstlisting}

\begin{lstlisting}

Set(A,n) // O(k)
	// PRE: A azzerato
	while i<=k and n>0
		A[i] = n%2
		n = n/2     // preso per difetto
		i++

\end{lstlisting}

Sia $\Phi(A)=$``numero di bit impostati a 1''. Il costo di una increment è $c=1+t$, dove $t\ge0$ è il numero di bit trasformati in 1 da 0. Sia $A_0$ lo stato iniziale del contatore azzerato ed $A_1$ il suo stato dopo aver eseguito una Set(A,n). Supponiamo di effettuare $m$ esecuzioni di increment per valutare l'analisi ammortizzata:

$$\Delta\Phi=\Phi(A_m)-\Phi(A_1)=t-1$$

Il numero di bit 1 rispetto ad $A_1$ varia di $-t+1$ in $A_m$. Dunque aggiungiamo $k$ alla formula per calcolare $\hat{c}$, distribuendolo a tutte le operazioni;

$$\hat{c}=c+\Phi(A_n)-\Phi(A_0)+\frac{k}{m}=1+t-t+1+\frac{k}{m}$$

Alla fine il costo ammortizzato è $O(1+\frac{k}{m})$, $m=\Omega(k), m\ge k$.

$$\frac{O(k)+O(1+\frac{k}{m})}{m+1}=O(\frac{k+1+k/m}{m+1})\le O(\frac{k+1+k/k}{k+1})=O(\frac{k+2}{k+1})=O(1)$$

\subsection{Esercizio 2}

Si vuole realizzare un timer usando un array $A$ per memorizzare i $k+1$ bit $b_k,....,b_1,b_0$ della rappresentazione binaria del valore del timer. Le operazioni previste per un timer sono \textit{Set(A,n)} che carica il timer ad $n$ secondi e \textit{Decrement(A)} che diminuisce di un secondo il valore del timer. L'operazione Set si può eseguire solamente quando il timer è azzerato mentre l'operazione Decrement si può eseguire soltanto quando il timer ha valore maggiore di 0. Scrivere le due funzioni Set e Decrement ed analizzarne la complessità ammortizzata.
\linebreak
\linebreak
\textbf{Suggerimento}: Memorizzare l'indice $m$ del bit 1 più significativo ($m=-1$ se tutti i bit sono 0, $m=k$ se tutti i bit sono a 1) ed usare come funzione potenziale il numero di bit uguali a 0 che precedono $A[m]$ più due volte il numero di bit che seguono $A[m]$ (che sono tutti 0), ossia $\Phi=\sum_{i=0}^{m-1}(1-b_i)+2(k-m)$.
\linebreak
\linebreak
\textbf{Soluzione}: Vediamo anzitutto gli pseudo-codici delle due operazioni:

\begin{lstlisting}

Set(A,n) // tutti i bit a 0, n<2^{k+1}
	k = A.lenght
	i = 0  // parto dal bit meno significativo
	while n>0 and i<=k
		A[i] = n%2  // il resto della divisione per 2
		n = n/2   // divisione intera per difetto
		i++
	if i>k
		error "underflow"
	else
		A.m = i-1

\end{lstlisting}

\begin{lstlisting}

Decrement(A)
	i = 0
	while A[i] == 0
		A[i] = 1
		i++
	A[i] = 0  // ora devo mettere apposto l'n
	if A.m == i
		A.m = A.m-1

\end{lstlisting}

\textbf{Analisi ammortizzata di Set}:

$$\hat{c}=c+\Delta\Phi$$
$$\Delta\Phi=\Phi+\Phi'=2(k-m)+\sum_{i=0}^{m-1}(1-b_i)-2(k+1)=$$
$$=2k-2m+\sum_{i=0}^{m-1}(1-b_i)-2k-2$$
$$\hat{c}\le m+1-2m+m-2 \le -1$$

Il costo è costante quindi ho concluso la dimostrazione.
\linebreak
\linebreak
\textbf{Analisi ammortizzata di Decrement}
\linebreak
\linebreak
$c=c+1$, proporzionale a $t-1$, dove $t$ è il numero di bit trasformati in 0.

$$\hat{c}=c+\Delta\Phi$$
$$\Delta\Phi=\Phi-\Phi'=2(k-m)+\sum_{i=0}^{m-1}(1-b_i)-2(k-m')+\sum_{i=0}^{m'-1}(1-b_i)=$$
$$-2m+2m'-t+1$$

$m$ ed $m'$ possono al massimo diversi tra loro di 1, quindi:

$$\Delta\Phi\le 2-t+1=3-t$$

$$\hat{c}=t+1+3-t=4$$

Ottengo un valore costante, inoltre:

$$\Phi_0=2(k+1)$$

In questo caso perchè il costo ammortizzato vada bene deve essere maggiore o uguale del costo reale.

Prendiamo le operazioni di segno:

$$\hat{c}_i=c_i+\Phi_i-\Phi_{i-1}$$
$$\sum_{i=1}^n\hat{c}_i=\sum_{i=1}^n\hat{c}_i+\Phi_n+\Phi_0\le\sum_{i=0}{n}(\hat{c}_i-\frac{\Phi_0}{n})$$
$$\hat{c}_i=c_i+\frac{\Phi_0}{n}$$

Devo distribuire, il costo ammortizzato è quindi:

$$\hat{c}=4+\frac{2(k+1)}{n}$$
Se $n=\Omega(k) \Rightarrow \hat{c}=O(1)$


%\appendix

%\input{glossario}


\end{document}