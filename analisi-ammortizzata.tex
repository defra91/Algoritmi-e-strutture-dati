\section{Analisi ammortizzata}

Si consideri il tempo richiesto per eseguire, nel \textbf{caso pessimo}, un'intera sequenza di operazioni. Se le operazioni costose sono relativamente meno frequenti allora il costo richiesto per eseguirle può essere ammortizzato con l'esecuzione delle operazioni meno costose.

\subsection{Metodo dell'aggregazione}

Si basa sul concetto di \textbf{costo ammortizzato}: data una sequenza di $n$ istruzioni aventi complessità $O(f(n))$, il costo ammortizzato della singola operazione si ottiene dividendo la complessità totale per il numero di istruzioni.

$$\hat{c}=\frac{O(f(n))}{n}$$

\subsection{Metodo degli accantonamenti (o dei crediti)}

Si caricano le operazioni meno costose di un costo aggiuntivo. Il costo aggiuntivo viene assegnato come \textbf{credito prepagato} a certi oggetti nella struttura dati. I crediti accumulati saranno usati per pagare le operazioni più costose su tali oggetti. 

Il costo ammortizzato delle operazioni meno costose è il costo effettivo aumentato del costo aggiuntivo. 

Il costo ammortizzato delle operazioni più costose è il costo effettivo diminuito del credito prepagato utilizzato.

Il costo artificiale fornisce un \textbf{limite superiore} al costo ammortizzato.

\subsection{Metodo del potenziale}

Si associa alla struttura dati $D$ un \textbf{potenziale} $\Phi(D)$ tale che la modifica della struttura dati dovuta alle operazioni meno costose comporti un aumento del potenziale, mentre le operazioni meno costose lo facciano diminuire.

Il costo ammortizzato è quindi la \textbf{somma algebrica} del costo effettivo e della variazione di potenziale. $D_i$ è la struttura dati dopo la \textit{c}-esima operazione e $c_i$ + il costo dell'\textit{i}-esima operazione.

$$\hat{c}_i=c_i+\Phi(D_i)-\Phi(D_{i-1})$$

Il costo ammortizzato di una sequenza di $n$ operazioni è:

$$\hat{c}=\sum_{i=1}^n\hat{c}_i=\sum_{i=1}^n[c_i+\Phi(D_i)-\Phi(D_{i-1})]$$
$$=c+\Phi(D_n)-\Phi(D_0)$$

Se la variazione $\Phi(D_n)-\Phi(D_0)$ del potenziale relativo all'esecuzione di tutta la sequenza non è negativa allora il costo ammortizzato $\hat{c}$ è una \textbf{maggiorazione} del costo reale $c$.

Altrimenti un valore $\Phi(D_n)-\Phi(D_0)$ negativo deve essere compensato con un aumento adeguato del costo ammortizzato delle operazioni.

\subsection{Esercizio 1}

Si vuole realizzare un contatore binario usando un array $A[0...k]$ per memorizzare i $k+1$ bit $b_k...b_1b_0$ della rappresentazione binaria del valore $x$ del contatore. Si vuole che il contatore possa iniziare anche con un valore maggiore di zero. A tale scopop si vuole che, oltre all'operazione \textbf{increment}, che aumenta di 1 il valore del contatore, sia prevista anche un'operazione iniziale \textbf{SET(A,n)}, che inizializza ad n il valore del contatore.
Usare il \textbf{metodo di aggregazione} per dimostrare che le operazioni di una sequenza costituita da una set seguita da $m$ increment, con $m=\Omega(k)$ hanno costo ammortizzato costante.
\linebreak
\linebreak
\textbf{Soluzione}:Vediamo anzitutto gli pseudo-codici delle due funzioni:

\begin{lstlisting}

Increment(A)
	i = 0
	while i<=k and A[i]==1
		A[i] = 0
		i++
	if i<=k
		A[i] = 1

\end{lstlisting}

\begin{lstlisting}

Set(A,n) // O(k)
	// PRE: A azzerato
	while i<=k and n>0
		A[i] = n%2
		n = n/2     // preso per difetto
		i++

\end{lstlisting}

Sia $\Phi(A)=$``numero di bit impostati a 1''. Il costo di una increment è $c=1+t$, dove $t\ge0$ è il numero di bit trasformati in 1 da 0. Sia $A_0$ lo stato iniziale del contatore azzerato ed $A_1$ il suo stato dopo aver eseguito una Set(A,n). Supponiamo di effettuare $m$ esecuzioni di increment per valutare l'analisi ammortizzata:

$$\Delta\Phi=\Phi(A_m)-\Phi(A_1)=t-1$$

Il numero di bit 1 rispetto ad $A_1$ varia di $-t+1$ in $A_m$. Dunque aggiungiamo $k$ alla formula per calcolare $\hat{c}$, distribuendolo a tutte le operazioni;

$$\hat{c}=c+\Phi(A_n)-\Phi(A_0)+\frac{k}{m}=1+t-t+1+\frac{k}{m}$$

Alla fine il costo ammortizzato è $O(1+\frac{k}{m})$, $m=\Omega(k), m\ge k$.

$$\frac{O(k)+O(1+\frac{k}{m})}{m+1}=O(\frac{k+1+k/m}{m+1})\le O(\frac{k+1+k/k}{k+1})=O(\frac{k+2}{k+1})=O(1)$$

\subsection{Esercizio 2}

Si vuole realizzare un timer usando un array $A$ per memorizzare i $k+1$ bit $b_k,....,b_1,b_0$ della rappresentazione binaria del valore del timer. Le operazioni previste per un timer sono \textit{Set(A,n)} che carica il timer ad $n$ secondi e \textit{Decrement(A)} che diminuisce di un secondo il valore del timer. L'operazione Set si può eseguire solamente quando il timer è azzerato mentre l'operazione Decrement si può eseguire soltanto quando il timer ha valore maggiore di 0. Scrivere le due funzioni Set e Decrement ed analizzarne la complessità ammortizzata.
\linebreak
\linebreak
\textbf{Suggerimento}: Memorizzare l'indice $m$ del bit 1 più significativo ($m=-1$ se tutti i bit sono 0, $m=k$ se tutti i bit sono a 1) ed usare come funzione potenziale il numero di bit uguali a 0 che precedono $A[m]$ più due volte il numero di bit che seguono $A[m]$ (che sono tutti 0), ossia $\Phi=\sum_{i=0}^{m-1}(1-b_i)+2(k-m)$.
\linebreak
\linebreak
\textbf{Soluzione}: Vediamo anzitutto gli pseudo-codici delle due operazioni:

\begin{lstlisting}

Set(A,n) // tutti i bit a 0, n<2^{k+1}
	k = A.lenght
	i = 0  // parto dal bit meno significativo
	while n>0 and i<=k
		A[i] = n%2  // il resto della divisione per 2
		n = n/2   // divisione intera per difetto
		i++
	if i>k
		error "underflow"
	else
		A.m = i-1

\end{lstlisting}

\begin{lstlisting}

Decrement(A)
	i = 0
	while A[i] == 0
		A[i] = 1
		i++
	A[i] = 0  // ora devo mettere apposto l'n
	if A.m == i
		A.m = A.m-1

\end{lstlisting}

\textbf{Analisi ammortizzata di Set}:

$$\hat{c}=c+\Delta\Phi$$
$$\Delta\Phi=\Phi+\Phi'=2(k-m)+\sum_{i=0}^{m-1}(1-b_i)-2(k+1)=$$
$$=2k-2m+\sum_{i=0}^{m-1}(1-b_i)-2k-2$$
$$\hat{c}\le m+1-2m+m-2 \le -1$$

Il costo è costante quindi ho concluso la dimostrazione.
\linebreak
\linebreak
\textbf{Analisi ammortizzata di Decrement}
\linebreak
\linebreak
$c=c+1$, proporzionale a $t-1$, dove $t$ è il numero di bit trasformati in 0.

$$\hat{c}=c+\Delta\Phi$$
$$\Delta\Phi=\Phi-\Phi'=2(k-m)+\sum_{i=0}^{m-1}(1-b_i)-2(k-m')+\sum_{i=0}^{m'-1}(1-b_i)=$$
$$-2m+2m'-t+1$$

$m$ ed $m'$ possono al massimo diversi tra loro di 1, quindi:

$$\Delta\Phi\le 2-t+1=3-t$$

$$\hat{c}=t+1+3-t=4$$

Ottengo un valore costante, inoltre:

$$\Phi_0=2(k+1)$$

In questo caso perchè il costo ammortizzato vada bene deve essere maggiore o uguale del costo reale.

Prendiamo le operazioni di segno:

$$\hat{c}_i=c_i+\Phi_i-\Phi_{i-1}$$
$$\sum_{i=1}^n\hat{c}_i=\sum_{i=1}^n\hat{c}_i+\Phi_n+\Phi_0\le\sum_{i=0}{n}(\hat{c}_i-\frac{\Phi_0}{n})$$
$$\hat{c}_i=c_i+\frac{\Phi_0}{n}$$

Devo distribuire, il costo ammortizzato è quindi:

$$\hat{c}=4+\frac{2(k+1)}{n}$$
Se $n=\Omega(k) \Rightarrow \hat{c}=O(1)$
