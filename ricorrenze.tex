\section{Risoluzione di ricorrenze}

\subsection{Metodo di sostituzione}

Per capire il metodo di sostituzione proviamo a risolvere il seguente esercizio:
\linebreak
\linebreak
La ricorrenza $T(n)=4T(n/2)+n^2\log n$ si può risolvere con il metodo dell'esperto? Giustificare la risposta. Se la risposta è negativa usare il metodo di sostituzione per dimostrare che $T(n)=O(n^2\log^2n)$.
\linebreak
\linebreak
Anzitutto vediamo i dati a disposizione:

$$a = 4, b = 2$$ 
$$f(n)=n^2\log n$$
$$g(n)=n^{\log_{b}a}=n^{\log_{2}4}=n^2$$

Calcoliamo ora il limite:

$$\lim_{n \to +\infty}\frac{n^2\log n}{n^2}=\infty$$

Da cui deduco che:

$$f(n)=\Omega(n^2)$$

Potrei dunque essere nel caso 3. Devo trovare un $$\epsilon > 0$$  tale che:

$$\lim_{n \to +\infty}\frac{n^2\log n}{n^{2+\epsilon}}\neq0$$

Ma mi accorgo subito che la cosa è impossibile, in quanto il denominatore, incrementando l'esponente, crescerà molto più velocemente rispetto al numeratore, per cui avrò sempre un valore tendente allo zero. Da questa considerazione deduco che la ricorrenza \textbf{non è risolvibile con il metodo dell'esperto}.
\linebreak
\linebreak
Procedo dunque con la sostituzione. Proviamo $T(n)=O(n^2\log^2n)$. Assumiamo che per un'opportuna costante $C>1$ e $\forall x<n$ sia verificata la disuguaglianza $T(x)\le C(x^2\log^2x)$ e dimostriamo che vale anche per $n$:

$$T(n)=4T(n/2)+n^2\log n \le 4C(n/2)^2\log^2(n/2)+n^2\log n$$
$$=Cn^2(\log n -1)^2+n^2\log n$$
$$=Cn^2(\log^2n-2\log n+1)+n^2\log n$$
$$=Cn^2\log^2n-2Cn^2\log n +Cn^2\log n+Cn^2+n^2\log n$$
$$=Cn^2\log^2n-(C-1)n^2\log n-Cn^2(\log n -1)$$

Ora applico una \textbf{maggiorazione}:

$$\le Cn^2\log^2n$$

Dunque ho dimostrato che: $T(n)=O(n^2\log^2n)$